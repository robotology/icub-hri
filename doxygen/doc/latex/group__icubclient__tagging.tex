\hypertarget{group__icubclient__tagging}{}\section{Run the proactive tagging demo}
\label{group__icubclient__tagging}\index{Run the proactive tagging demo@{Run the proactive tagging demo}}


This provides a brief tutorial on how to run the proactive tagging demo described in the paper.  


This provides a brief tutorial on how to run the proactive tagging demo described in the paper. 

To run the proactive tagging demo, first install all modules as described in the R\+E\+A\+D\+ME. Then, adapt the {\ttfamily proactive\+Tagging.\+xml} within {\ttfamily icub-\/client/app/demos} to match your system setup (modify the node names). For the role of the individual components, we refer to our paper.

Once you start all modules, two drives will be active as specified in {\ttfamily icub-\/client/app/reactive\+Layer/allostatic\+Controller/conf/default.\+ini}\+: one for knowledge acquisition, and another one for knowledge demonstration. Both drives will start decaying (see \hyperlink{group__homeostasis_classHomeostaticModule}{Homeostatic\+Module} and \hyperlink{group__allostaticController_a448eb3497467e10577e6a03ef55633f3}{Allostatic\+Controller\+::update\+Allostatic()}). Once the drive for knowledge acquisition hits the lower threshold, the \hyperlink{group__allostaticController_a70e9a461682da34e29e77d5d85d5a11c}{Allostatic\+Drive\+::trigger\+Behavior()} is being called, which in turn sends an R\+PC command to the \hyperlink{group__behaviorManager_classBehaviorManager}{Behavior\+Manager}. Within \hyperlink{group__behaviorManager_a79cbdf143299a44a1c391e385846bb04}{Behavior\+Manager\+::respond()}, the \hyperlink{group__behaviorManager_a8427d7f479b580bbf54030015b506374}{Tagging\+::run()} is called.

There, it is specified that the i\+Cub should first find the name of the partner. Therefore, an R\+PC command is send to \hyperlink{classproactiveTagging}{proactive\+Tagging}, which where the \hyperlink{classproactiveTagging_a7645a181289f171fc5268ecd24e7ace9}{proactive\+Tagging\+::explore\+Unknown\+Entity()} method is called. This method uses the face recognition ability of \hyperlink{namespaceSAM}{S\+AM} (see \hyperlink{group__icubclient__SAM__Drivers_classSAM_1_1SAM__Drivers_1_1SAMDriver__interaction_1_1SAMDriver__interaction}{S\+A\+M.\+S\+A\+M\+\_\+\+Drivers.\+S\+A\+M\+Driver\+\_\+interaction.\+S\+A\+M\+Driver\+\_\+interaction}). If the face is not being recognized, it will ask the partner for the name (see \hyperlink{classproactiveTagging_a7645a181289f171fc5268ecd24e7ace9}{proactive\+Tagging\+::explore\+Unknown\+Entity()} and \hyperlink{classproactiveTagging_a7d77039a8eb90ea145043d28a2b5435c}{proactive\+Tagging\+::get\+Name\+From\+S\+A\+M()}). Finally, the I\+Cub\+Client\+::change\+Name() method is called which changes the name in the working memory and informs other modules of the name change.

The next times the threshold is hit, the i\+Cub will either ask for the name of an object by pointing to it, or ask for the name of the body part by moving it. You can use speech input to provide the name of the object / body part. The possible list of (object, bodypart and agent) names is coded within the X\+ML files in the {\ttfamily icub-\/client/app/proactive\+Tagging/conf} directory. Also, if the name of a body part is known but its corresponding tactile patch is not known yet, the i\+Cub will move the body part and ask you to touch the corresponding part (see \hyperlink{classproactiveTagging_ad767fe5d8389c2773da1d34501b8b792}{proactive\+Tagging\+::explore\+Tactile\+Entity\+With\+Name()}). The logic for tagging subsequent objects and body parts is the same as introduced above for naming the partner.

Besides waiting for the drives to decay, you can also give commands to the i\+Cub. The speech commands are recognized within \hyperlink{group__ears_ae6ddc924bc901594ab56603aacb2b95f}{ears\+::update\+Module()} that calls the proper module according to the given command. You can say\+: \char`\"{}\+Please take the X\char`\"{} (pulling an object), \char`\"{}\+Give me the X\char`\"{} (pushing an object), \char`\"{}\+Look at the X\char`\"{}, \char`\"{}\+Point the X\char`\"{}, where X is the name of an object. The names are coded in the {\ttfamily icub-\/client/app/ears/conf/\+Main\+Grammar.\+xml} file. We recommend keeping these object names consistent with the names in {\ttfamily icub-\/client/app/proactive\+Tagging/conf}. There is some overhead involved when giving commands to the i\+Cub, as every time the \hyperlink{group__behaviorManager_a8427d7f479b580bbf54030015b506374}{Tagging\+::run()} method is being called. This method does nothing in case the object name is known within the working memory, or calls \hyperlink{classproactiveTagging_a06f3da8e08290cc4e667dc1f1ab0a2ba}{proactive\+Tagging\+::searching\+Entity()} to find which object the human is referring to. If there is only one unknown object, the robot considers that to be the one the human referred to. If there are multiple unknown objects, the i\+Cub will instead try to infer which object is known by recognising a pointing action of the partner.

Once the partner has pointed to the object, the i\+Cub knows the name and can thus execute the requested action. The mapping from command to action is defined in \hyperlink{group__ears_ae6ddc924bc901594ab56603aacb2b95f}{ears\+::update\+Module()}. In case an object should be pulled or pushed, the \hyperlink{group__behaviorManager_a2e242410f34618b2ff73f9106732470b}{Move\+Object\+::run()} method is called. When asked to recognize an action, the \hyperlink{group__behaviorManager_a0a095aa1af770bac92f30042cee21e3a}{Recognition\+Order\+::run()} is called. Finally, the \hyperlink{group__behaviorManager_a6f1a7b52f3c846009e6b510dbb728011}{Pointing\+::run()} method takes care of pointing to an object when ordered to do so.

(This page can be edited at \hyperlink{proactivetagging_8dox}{src/doc/proactivetagging.\+dox}) 